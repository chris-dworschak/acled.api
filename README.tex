% Options for packages loaded elsewhere
\PassOptionsToPackage{unicode}{hyperref}
\PassOptionsToPackage{hyphens}{url}
%
\documentclass[
]{article}
\usepackage{lmodern}
\usepackage{amsmath}
\usepackage{ifxetex,ifluatex}
\ifnum 0\ifxetex 1\fi\ifluatex 1\fi=0 % if pdftex
  \usepackage[T1]{fontenc}
  \usepackage[utf8]{inputenc}
  \usepackage{textcomp} % provide euro and other symbols
  \usepackage{amssymb}
\else % if luatex or xetex
  \usepackage{unicode-math}
  \defaultfontfeatures{Scale=MatchLowercase}
  \defaultfontfeatures[\rmfamily]{Ligatures=TeX,Scale=1}
\fi
% Use upquote if available, for straight quotes in verbatim environments
\IfFileExists{upquote.sty}{\usepackage{upquote}}{}
\IfFileExists{microtype.sty}{% use microtype if available
  \usepackage[]{microtype}
  \UseMicrotypeSet[protrusion]{basicmath} % disable protrusion for tt fonts
}{}
\makeatletter
\@ifundefined{KOMAClassName}{% if non-KOMA class
  \IfFileExists{parskip.sty}{%
    \usepackage{parskip}
  }{% else
    \setlength{\parindent}{0pt}
    \setlength{\parskip}{6pt plus 2pt minus 1pt}}
}{% if KOMA class
  \KOMAoptions{parskip=half}}
\makeatother
\usepackage{xcolor}
\IfFileExists{xurl.sty}{\usepackage{xurl}}{} % add URL line breaks if available
\IfFileExists{bookmark.sty}{\usepackage{bookmark}}{\usepackage{hyperref}}
\hypersetup{
  hidelinks,
  pdfcreator={LaTeX via pandoc}}
\urlstyle{same} % disable monospaced font for URLs
\usepackage[margin=1in]{geometry}
\usepackage{color}
\usepackage{fancyvrb}
\newcommand{\VerbBar}{|}
\newcommand{\VERB}{\Verb[commandchars=\\\{\}]}
\DefineVerbatimEnvironment{Highlighting}{Verbatim}{commandchars=\\\{\}}
% Add ',fontsize=\small' for more characters per line
\usepackage{framed}
\definecolor{shadecolor}{RGB}{248,248,248}
\newenvironment{Shaded}{\begin{snugshade}}{\end{snugshade}}
\newcommand{\AlertTok}[1]{\textcolor[rgb]{0.94,0.16,0.16}{#1}}
\newcommand{\AnnotationTok}[1]{\textcolor[rgb]{0.56,0.35,0.01}{\textbf{\textit{#1}}}}
\newcommand{\AttributeTok}[1]{\textcolor[rgb]{0.77,0.63,0.00}{#1}}
\newcommand{\BaseNTok}[1]{\textcolor[rgb]{0.00,0.00,0.81}{#1}}
\newcommand{\BuiltInTok}[1]{#1}
\newcommand{\CharTok}[1]{\textcolor[rgb]{0.31,0.60,0.02}{#1}}
\newcommand{\CommentTok}[1]{\textcolor[rgb]{0.56,0.35,0.01}{\textit{#1}}}
\newcommand{\CommentVarTok}[1]{\textcolor[rgb]{0.56,0.35,0.01}{\textbf{\textit{#1}}}}
\newcommand{\ConstantTok}[1]{\textcolor[rgb]{0.00,0.00,0.00}{#1}}
\newcommand{\ControlFlowTok}[1]{\textcolor[rgb]{0.13,0.29,0.53}{\textbf{#1}}}
\newcommand{\DataTypeTok}[1]{\textcolor[rgb]{0.13,0.29,0.53}{#1}}
\newcommand{\DecValTok}[1]{\textcolor[rgb]{0.00,0.00,0.81}{#1}}
\newcommand{\DocumentationTok}[1]{\textcolor[rgb]{0.56,0.35,0.01}{\textbf{\textit{#1}}}}
\newcommand{\ErrorTok}[1]{\textcolor[rgb]{0.64,0.00,0.00}{\textbf{#1}}}
\newcommand{\ExtensionTok}[1]{#1}
\newcommand{\FloatTok}[1]{\textcolor[rgb]{0.00,0.00,0.81}{#1}}
\newcommand{\FunctionTok}[1]{\textcolor[rgb]{0.00,0.00,0.00}{#1}}
\newcommand{\ImportTok}[1]{#1}
\newcommand{\InformationTok}[1]{\textcolor[rgb]{0.56,0.35,0.01}{\textbf{\textit{#1}}}}
\newcommand{\KeywordTok}[1]{\textcolor[rgb]{0.13,0.29,0.53}{\textbf{#1}}}
\newcommand{\NormalTok}[1]{#1}
\newcommand{\OperatorTok}[1]{\textcolor[rgb]{0.81,0.36,0.00}{\textbf{#1}}}
\newcommand{\OtherTok}[1]{\textcolor[rgb]{0.56,0.35,0.01}{#1}}
\newcommand{\PreprocessorTok}[1]{\textcolor[rgb]{0.56,0.35,0.01}{\textit{#1}}}
\newcommand{\RegionMarkerTok}[1]{#1}
\newcommand{\SpecialCharTok}[1]{\textcolor[rgb]{0.00,0.00,0.00}{#1}}
\newcommand{\SpecialStringTok}[1]{\textcolor[rgb]{0.31,0.60,0.02}{#1}}
\newcommand{\StringTok}[1]{\textcolor[rgb]{0.31,0.60,0.02}{#1}}
\newcommand{\VariableTok}[1]{\textcolor[rgb]{0.00,0.00,0.00}{#1}}
\newcommand{\VerbatimStringTok}[1]{\textcolor[rgb]{0.31,0.60,0.02}{#1}}
\newcommand{\WarningTok}[1]{\textcolor[rgb]{0.56,0.35,0.01}{\textbf{\textit{#1}}}}
\usepackage{graphicx}
\makeatletter
\def\maxwidth{\ifdim\Gin@nat@width>\linewidth\linewidth\else\Gin@nat@width\fi}
\def\maxheight{\ifdim\Gin@nat@height>\textheight\textheight\else\Gin@nat@height\fi}
\makeatother
% Scale images if necessary, so that they will not overflow the page
% margins by default, and it is still possible to overwrite the defaults
% using explicit options in \includegraphics[width, height, ...]{}
\setkeys{Gin}{width=\maxwidth,height=\maxheight,keepaspectratio}
% Set default figure placement to htbp
\makeatletter
\def\fps@figure{htbp}
\makeatother
\setlength{\emergencystretch}{3em} % prevent overfull lines
\providecommand{\tightlist}{%
  \setlength{\itemsep}{0pt}\setlength{\parskip}{0pt}}
\setcounter{secnumdepth}{-\maxdimen} % remove section numbering
\ifluatex
  \usepackage{selnolig}  % disable illegal ligatures
\fi

\author{}
\date{\vspace{-2.5em}}

\begin{document}

\hypertarget{automated-retrieval-of-acled-conflict-event-data}{%
\section{Automated Retrieval of ACLED Conflict Event
Data}\label{automated-retrieval-of-acled-conflict-event-data}}

\href{https://CRAN.R-project.org/package=acled.api/}{\includegraphics{https://www.r-pkg.org/badges/version-last-release/acled.api}}
\href{https://lifecycle.r-lib.org/articles/stages.html\#stable/}{\includegraphics{https://img.shields.io/badge/lifecycle-stable-brightgreen.svg}}
\href{https://travis-ci.com/gitlab/chris-dworschak/acled.api/}{\includegraphics{https://travis-ci.com/chris-dworschak/acled.api.svg?branch=master}}
\href{https://CRAN.R-project.org/package=acled.api/}{\includegraphics{http://cranlogs.r-pkg.org/badges/grand-total/acled.api}}

This small package provides functionality to access and manage the
application programming interface (API) of the
\href{https://acleddata.com/}{Armed Conflict Location \& Event Data
Project (ACLED)}, while requiring a minimal number of dependencies. The
function \texttt{acled.api()} makes it easy to retrieve a user-defined
sample (or all of the available data) of ACLED, enabling a seamless
integration of regular data updates into the research work flow.

When using this package, you acknowledge that you have read ACLED's
terms and conditions of use, and that you agree with their attribution
requirements.

\hypertarget{installation}{%
\subsection{Installation}\label{installation}}

You can install the latest release version of acled.api from
\href{https://CRAN.R-project.org/package=acled.api/}{CRAN} with:

\begin{Shaded}
\begin{Highlighting}[]
\FunctionTok{install.packages}\NormalTok{(}\StringTok{"acled.api"}\NormalTok{) }\CommentTok{\# downloads and installs the package from CRAN}
\end{Highlighting}
\end{Shaded}

You can install the development version from
\href{https://gitlab.com/chris-dworschak/}{GitLab} with:

\begin{Shaded}
\begin{Highlighting}[]
\NormalTok{remotes}\SpecialCharTok{::}\FunctionTok{install\_gitlab}\NormalTok{(}\StringTok{"chris{-}dworschak/acled.api"}\NormalTok{) }\CommentTok{\# downloads and installs the package from GitLab}
\end{Highlighting}
\end{Shaded}

\hypertarget{example}{%
\subsection{Example}\label{example}}

Using \texttt{acled.api} is straight forward. To download data on, for
example, all ACLED conflict events in Europe and Central America that
happened between June 2019 and July 2020, you can supply:

\begin{Shaded}
\begin{Highlighting}[]
\FunctionTok{library}\NormalTok{(acled.api) }\CommentTok{\# loads the package}
\CommentTok{\#\textgreater{} }
\CommentTok{\#\textgreater{} By using this package, you acknowledge that you have read ACLED\textquotesingle{}s terms and}
\CommentTok{\#\textgreater{} conditions. The data must be cited as per ACLED attribution requirements. To}
\CommentTok{\#\textgreater{} download ACLED data, you require an ACLED access key. You can request your key}
\CommentTok{\#\textgreater{} by freely registering with ACLED on https://developer.acleddata.com/.}
\CommentTok{\#\textgreater{} The package may be cited as:}
\CommentTok{\#\textgreater{} Dworschak, Christoph. 2020. "Acled.api: Automated Retrieval of ACLED Conflict}
\CommentTok{\#\textgreater{} Event Data." R package. CRAN version 1.1.0.}
\CommentTok{\#\textgreater{} For the development version of this package, visit \textless{}https://gitlab.com/chris{-}dworschak/acled.api/\textgreater{}}

\NormalTok{my.data.frame }\OtherTok{\textless{}{-}} \FunctionTok{acled.api}\NormalTok{( }\CommentTok{\# stores an ACLED sample in object my.data.frame}
  \AttributeTok{email.address =} \FunctionTok{Sys.getenv}\NormalTok{(}\StringTok{"EMAIL\_ADDRESS"}\NormalTok{),}
  \AttributeTok{access.key =} \FunctionTok{Sys.getenv}\NormalTok{(}\StringTok{"ACCESS\_KEY"}\NormalTok{),}
  \AttributeTok{region =} \FunctionTok{c}\NormalTok{(}\StringTok{"Southern Asia"}\NormalTok{, }\StringTok{"Central America"}\NormalTok{), }
  \AttributeTok{start.date =} \StringTok{"2019{-}06{-}01"}\NormalTok{, }
  \AttributeTok{end.date =} \StringTok{"2020{-}07{-}31"}\NormalTok{)}
\CommentTok{\#\textgreater{} Your ACLED data request was successful. }
\CommentTok{\#\textgreater{} Events were retrieved for the period starting 2019{-}06{-}01 until 2020{-}07{-}31.}

\NormalTok{my.data.frame[}\DecValTok{1}\SpecialCharTok{:}\DecValTok{3}\NormalTok{,] }\CommentTok{\# returns the first three observations of the ACLED sample}
\CommentTok{\#\textgreater{}                      region     country year event\_date}
\CommentTok{\#\textgreater{} 1 Caucasus and Central Asia Afghanistan 2020 2020{-}07{-}31}
\CommentTok{\#\textgreater{} 2 Caucasus and Central Asia  Azerbaijan 2020 2020{-}07{-}31}
\CommentTok{\#\textgreater{} 3 Caucasus and Central Asia Afghanistan 2020 2020{-}07{-}31}
\CommentTok{\#\textgreater{}                              source   admin1 admin2 admin3          location}
\CommentTok{\#\textgreater{} 1  Afghan Islamic Press News Agency   Ghazni  Andar                    Miray}
\CommentTok{\#\textgreater{} 2 Ministry of Defence of Azerbaijan Jabrayil               Chodjuk{-}Mardjanli}
\CommentTok{\#\textgreater{} 3  Afghan Islamic Press News Agency  Helmand Sangin                   Sangin}
\CommentTok{\#\textgreater{}                   event\_type               sub\_event\_type interaction}
\CommentTok{\#\textgreater{} 1     Strategic developments Looting/property destruction          37}
\CommentTok{\#\textgreater{} 2                    Battles                  Armed clash          11}
\CommentTok{\#\textgreater{} 3 Violence against civilians                       Attack          37}
\CommentTok{\#\textgreater{}   fatalities  timestamp}
\CommentTok{\#\textgreater{} 1          0 1607974383}
\CommentTok{\#\textgreater{} 2          0 1596473349}
\CommentTok{\#\textgreater{} 3          1 1599503174}
\end{Highlighting}
\end{Shaded}

\hypertarget{a-note-on-replicability}{%
\subsection{A note on replicability}\label{a-note-on-replicability}}

Some tasks, like real-time analyses and continuously updated forecasting
models (e.g., as used by practitioners), may not require replicability
of results. However, most research-related tasks assume the possibility
of replication at a later stage (e.g., analyses of which the results are
intended for publication, or a data project taking multiple days where a
change to the underlying sample is not desirable). After the release of
versions 1 through 8, ACLED changed their update system to allow for
real-time amendments and post-release corrections, thereby forgoing
traditional data versioning. This change requires researchers to take
additional steps in order to ensure the replicability of their results
when using ACLED data.

Importantly, downloaded data intended for replicable use must be
permanently stored by the analyst. Data downloaded through
\texttt{acled.api()} are only stored temporarily in the working space,
and may be lost after closing R. Therefore, if replicability is
important to the analyst's task, a call through \texttt{acled.api()}
should occur only once at the beginning of the data project, immediately
followed by, e.g.,
\texttt{saveRDS(downloaded.data,\ file\ =\ "my\_acled\_data.rds")}. This
locally stored data file can then be used again at a later point by
calling \texttt{readRDS(file\ =\ "my\_acled\_data.rds")}, and ensures
that the analysis sample stays constant over time.

ACLED provides a time stamp for each individual observation, enabling
researchers to do ``micro versioning'' of data points if necessary, and
to verify congruence across samples. For this it is important that
researchers do not drop the variable \emph{timestamp} during the data
management process. Starting version 1.0.9 the function
\texttt{acled.api()} includes the \emph{timestamp} variable in its
default API call.

\end{document}
